\documentclass[]{article}
\usepackage{lmodern}
\usepackage{amssymb,amsmath}
\usepackage{ifxetex,ifluatex}
\usepackage{fixltx2e} % provides \textsubscript
\ifnum 0\ifxetex 1\fi\ifluatex 1\fi=0 % if pdftex
  \usepackage[T1]{fontenc}
  \usepackage[utf8]{inputenc}
\else % if luatex or xelatex
  \ifxetex
    \usepackage{mathspec}
  \else
    \usepackage{fontspec}
  \fi
  \defaultfontfeatures{Ligatures=TeX,Scale=MatchLowercase}
\fi
% use upquote if available, for straight quotes in verbatim environments
\IfFileExists{upquote.sty}{\usepackage{upquote}}{}
% use microtype if available
\IfFileExists{microtype.sty}{%
\usepackage{microtype}
\UseMicrotypeSet[protrusion]{basicmath} % disable protrusion for tt fonts
}{}
\usepackage[margin=1in]{geometry}
\usepackage{hyperref}
\hypersetup{unicode=true,
            pdftitle={Finite Sample Property of MLE},
            pdfauthor={Nao Mimoto - Dept. of Statistics : The University of Akron},
            pdfborder={0 0 0},
            breaklinks=true}
\urlstyle{same}  % don't use monospace font for urls
\usepackage{color}
\usepackage{fancyvrb}
\newcommand{\VerbBar}{|}
\newcommand{\VERB}{\Verb[commandchars=\\\{\}]}
\DefineVerbatimEnvironment{Highlighting}{Verbatim}{commandchars=\\\{\}}
% Add ',fontsize=\small' for more characters per line
\usepackage{framed}
\definecolor{shadecolor}{RGB}{248,248,248}
\newenvironment{Shaded}{\begin{snugshade}}{\end{snugshade}}
\newcommand{\AlertTok}[1]{\textcolor[rgb]{0.94,0.16,0.16}{#1}}
\newcommand{\AnnotationTok}[1]{\textcolor[rgb]{0.56,0.35,0.01}{\textbf{\textit{#1}}}}
\newcommand{\AttributeTok}[1]{\textcolor[rgb]{0.77,0.63,0.00}{#1}}
\newcommand{\BaseNTok}[1]{\textcolor[rgb]{0.00,0.00,0.81}{#1}}
\newcommand{\BuiltInTok}[1]{#1}
\newcommand{\CharTok}[1]{\textcolor[rgb]{0.31,0.60,0.02}{#1}}
\newcommand{\CommentTok}[1]{\textcolor[rgb]{0.56,0.35,0.01}{\textit{#1}}}
\newcommand{\CommentVarTok}[1]{\textcolor[rgb]{0.56,0.35,0.01}{\textbf{\textit{#1}}}}
\newcommand{\ConstantTok}[1]{\textcolor[rgb]{0.00,0.00,0.00}{#1}}
\newcommand{\ControlFlowTok}[1]{\textcolor[rgb]{0.13,0.29,0.53}{\textbf{#1}}}
\newcommand{\DataTypeTok}[1]{\textcolor[rgb]{0.13,0.29,0.53}{#1}}
\newcommand{\DecValTok}[1]{\textcolor[rgb]{0.00,0.00,0.81}{#1}}
\newcommand{\DocumentationTok}[1]{\textcolor[rgb]{0.56,0.35,0.01}{\textbf{\textit{#1}}}}
\newcommand{\ErrorTok}[1]{\textcolor[rgb]{0.64,0.00,0.00}{\textbf{#1}}}
\newcommand{\ExtensionTok}[1]{#1}
\newcommand{\FloatTok}[1]{\textcolor[rgb]{0.00,0.00,0.81}{#1}}
\newcommand{\FunctionTok}[1]{\textcolor[rgb]{0.00,0.00,0.00}{#1}}
\newcommand{\ImportTok}[1]{#1}
\newcommand{\InformationTok}[1]{\textcolor[rgb]{0.56,0.35,0.01}{\textbf{\textit{#1}}}}
\newcommand{\KeywordTok}[1]{\textcolor[rgb]{0.13,0.29,0.53}{\textbf{#1}}}
\newcommand{\NormalTok}[1]{#1}
\newcommand{\OperatorTok}[1]{\textcolor[rgb]{0.81,0.36,0.00}{\textbf{#1}}}
\newcommand{\OtherTok}[1]{\textcolor[rgb]{0.56,0.35,0.01}{#1}}
\newcommand{\PreprocessorTok}[1]{\textcolor[rgb]{0.56,0.35,0.01}{\textit{#1}}}
\newcommand{\RegionMarkerTok}[1]{#1}
\newcommand{\SpecialCharTok}[1]{\textcolor[rgb]{0.00,0.00,0.00}{#1}}
\newcommand{\SpecialStringTok}[1]{\textcolor[rgb]{0.31,0.60,0.02}{#1}}
\newcommand{\StringTok}[1]{\textcolor[rgb]{0.31,0.60,0.02}{#1}}
\newcommand{\VariableTok}[1]{\textcolor[rgb]{0.00,0.00,0.00}{#1}}
\newcommand{\VerbatimStringTok}[1]{\textcolor[rgb]{0.31,0.60,0.02}{#1}}
\newcommand{\WarningTok}[1]{\textcolor[rgb]{0.56,0.35,0.01}{\textbf{\textit{#1}}}}
\usepackage{graphicx}
% grffile has become a legacy package: https://ctan.org/pkg/grffile
\IfFileExists{grffile.sty}{%
\usepackage{grffile}
}{}
\makeatletter
\def\maxwidth{\ifdim\Gin@nat@width>\linewidth\linewidth\else\Gin@nat@width\fi}
\def\maxheight{\ifdim\Gin@nat@height>\textheight\textheight\else\Gin@nat@height\fi}
\makeatother
% Scale images if necessary, so that they will not overflow the page
% margins by default, and it is still possible to overwrite the defaults
% using explicit options in \includegraphics[width, height, ...]{}
\setkeys{Gin}{width=\maxwidth,height=\maxheight,keepaspectratio}
\IfFileExists{parskip.sty}{%
\usepackage{parskip}
}{% else
\setlength{\parindent}{0pt}
\setlength{\parskip}{6pt plus 2pt minus 1pt}
}
\setlength{\emergencystretch}{3em}  % prevent overfull lines
\providecommand{\tightlist}{%
  \setlength{\itemsep}{0pt}\setlength{\parskip}{0pt}}
\setcounter{secnumdepth}{0}
% Redefines (sub)paragraphs to behave more like sections
\ifx\paragraph\undefined\else
\let\oldparagraph\paragraph
\renewcommand{\paragraph}[1]{\oldparagraph{#1}\mbox{}}
\fi
\ifx\subparagraph\undefined\else
\let\oldsubparagraph\subparagraph
\renewcommand{\subparagraph}[1]{\oldsubparagraph{#1}\mbox{}}
\fi

%%% Use protect on footnotes to avoid problems with footnotes in titles
\let\rmarkdownfootnote\footnote%
\def\footnote{\protect\rmarkdownfootnote}

%%% Change title format to be more compact
\usepackage{titling}

% Create subtitle command for use in maketitle
\providecommand{\subtitle}[1]{
  \posttitle{
    \begin{center}\large#1\end{center}
    }
}

\setlength{\droptitle}{-2em}

  \title{Finite Sample Property of MLE}
    \pretitle{\vspace{\droptitle}\centering\huge}
  \posttitle{\par}
    \author{\href{https://nmimoto.github.io/}{Nao Mimoto} -
\href{http://www.uakron.edu/stat/}{Dept. of Statistics} :
\href{http://www.uakron.edu}{The University of Akron}}
    \preauthor{\centering\large\emph}
  \postauthor{\par}
    \date{}
    \predate{}\postdate{}
  

\begin{document}
\maketitle

{
\setcounter{tocdepth}{2}
\tableofcontents
}
\href{https://nmimoto.github.io/477/}{TS Class Web Page} --
\href{https://nmimoto.github.io/R/}{R resource page}

\hypertarget{finite-sample-property-of-mle}{%
\section{1. Finite Sample Property of
MLE}\label{finite-sample-property-of-mle}}

\hypertarget{a.-1-time-simulation-and-mle}{%
\subsection{a. 1-time simulation and
MLE}\label{a.-1-time-simulation-and-mle}}

\begin{Shaded}
\begin{Highlighting}[]
\NormalTok{  mu <-}\StringTok{ }\DecValTok{5}
\NormalTok{  X  <-}\StringTok{ }\KeywordTok{arima.sim}\NormalTok{(}\DataTypeTok{n =} \DecValTok{200}\NormalTok{, }\KeywordTok{list}\NormalTok{(}\DataTypeTok{ar =} \KeywordTok{c}\NormalTok{(}\FloatTok{0.7}\NormalTok{), }\DataTypeTok{ma =} \KeywordTok{c}\NormalTok{(}\FloatTok{0.4}\NormalTok{)) ) }\OperatorTok{+}\StringTok{ }\NormalTok{mu}
  \KeywordTok{plot}\NormalTok{(X, }\DataTypeTok{type=}\StringTok{"o"}\NormalTok{)}
\end{Highlighting}
\end{Shaded}

\includegraphics{TS-Lec53_files/figure-latex/unnamed-chunk-1-1.pdf}

\begin{Shaded}
\begin{Highlighting}[]
\NormalTok{  Fit1 <-}\StringTok{ }\KeywordTok{arima}\NormalTok{(X, }\DataTypeTok{order=}\KeywordTok{c}\NormalTok{(}\DecValTok{1}\NormalTok{,}\DecValTok{0}\NormalTok{,}\DecValTok{1}\NormalTok{))}
\NormalTok{  Fit1}
\end{Highlighting}
\end{Shaded}

\begin{verbatim}
## 
## Call:
## arima(x = X, order = c(1, 0, 1))
## 
## Coefficients:
##          ar1     ma1  intercept
##       0.7657  0.4048     5.1084
## s.e.  0.0492  0.0656     0.3995
## 
## sigma^2 estimated as 0.9194:  log likelihood = -276.19,  aic = 560.37
\end{verbatim}

\begin{Shaded}
\begin{Highlighting}[]
  \KeywordTok{str}\NormalTok{(Fit1)          }\CommentTok{# see what's inside}
\end{Highlighting}
\end{Shaded}

\begin{verbatim}
## List of 14
##  $ coef     : Named num [1:3] 0.766 0.405 5.108
##   ..- attr(*, "names")= chr [1:3] "ar1" "ma1" "intercept"
##  $ sigma2   : num 0.919
##  $ var.coef : num [1:3, 1:3] 0.002419 -0.001339 0.000348 -0.001339 0.004302 ...
##   ..- attr(*, "dimnames")=List of 2
##   .. ..$ : chr [1:3] "ar1" "ma1" "intercept"
##   .. ..$ : chr [1:3] "ar1" "ma1" "intercept"
##  $ mask     : logi [1:3] TRUE TRUE TRUE
##  $ loglik   : num -276
##  $ aic      : num 560
##  $ arma     : int [1:7] 1 1 0 0 1 0 0
##  $ residuals: Time-Series [1:200] from 1 to 200: 0.511 0.285 -0.576 -0.126 0.533 ...
##  $ call     : language arima(x = X, order = c(1, 0, 1))
##  $ series   : chr "X"
##  $ code     : int 0
##  $ n.cond   : int 0
##  $ nobs     : int 200
##  $ model    :List of 10
##   ..$ phi  : num 0.766
##   ..$ theta: num 0.405
##   ..$ Delta: num(0) 
##   ..$ Z    : num [1:2] 1 0
##   ..$ a    : num [1:2] 0.7611 -0.0858
##   ..$ P    : num [1:2, 1:2] 0 0 0 0
##   ..$ T    : num [1:2, 1:2] 0.766 0 1 0
##   ..$ V    : num [1:2, 1:2] 1 0.405 0.405 0.164
##   ..$ h    : num 0
##   ..$ Pn   : num [1:2, 1:2] 1 0.405 0.405 0.164
##  - attr(*, "class")= chr "Arima"
\end{verbatim}

\begin{Shaded}
\begin{Highlighting}[]
\NormalTok{  Fit1}\OperatorTok{$}\NormalTok{coef          }\CommentTok{# parameter estimates (MLE)}
\end{Highlighting}
\end{Shaded}

\begin{verbatim}
##       ar1       ma1 intercept 
## 0.7656605 0.4047857 5.1084293
\end{verbatim}

\begin{Shaded}
\begin{Highlighting}[]
\NormalTok{  Fit1}\OperatorTok{$}\NormalTok{var.coef      }\CommentTok{# variance of MLE using asymptotic formula}
\end{Highlighting}
\end{Shaded}

\begin{verbatim}
##                     ar1           ma1     intercept
## ar1        0.0024189834 -0.0013390820  0.0003483548
## ma1       -0.0013390820  0.0043016667 -0.0002405917
## intercept  0.0003483548 -0.0002405917  0.1595776591
\end{verbatim}

\begin{Shaded}
\begin{Highlighting}[]
  \KeywordTok{c}\NormalTok{(Fit1}\OperatorTok{$}\NormalTok{var.coef[}\DecValTok{1}\NormalTok{,}\DecValTok{1}\NormalTok{], Fit1}\OperatorTok{$}\NormalTok{var.coef[}\DecValTok{2}\NormalTok{,}\DecValTok{2}\NormalTok{], Fit1}\OperatorTok{$}\NormalTok{var.coef[}\DecValTok{3}\NormalTok{,}\DecValTok{3}\NormalTok{])}
\end{Highlighting}
\end{Shaded}

\begin{verbatim}
## [1] 0.002418983 0.004301667 0.159577659
\end{verbatim}

\hypertarget{b.-put-above-in-a-loop-of-1000.}{%
\subsection{b. Put above in a loop of
1000.}\label{b.-put-above-in-a-loop-of-1000.}}

\begin{Shaded}
\begin{Highlighting}[]
  \CommentTok{#-Repeat above for 1000 times, recording Est$coef each time.}
  \CommentTok{#       Compare the simulated variance and theoretical asympt variance ---}

\NormalTok{  n=}\DecValTok{100}

\NormalTok{  MLE  <-}\StringTok{ }\KeywordTok{matrix}\NormalTok{(}\DecValTok{0}\NormalTok{,}\DecValTok{1000}\NormalTok{,}\DecValTok{7}\NormalTok{)  }\CommentTok{#matrix to record estimated values}
\NormalTok{  Vars <-}\StringTok{ }\KeywordTok{matrix}\NormalTok{(}\DecValTok{0}\NormalTok{,}\DecValTok{1000}\NormalTok{,}\DecValTok{7}\NormalTok{)  }\CommentTok{#matrix to record estimated values}

  \KeywordTok{set.seed}\NormalTok{(}\DecValTok{23373}\NormalTok{)}
  \ControlFlowTok{for}\NormalTok{ (i }\ControlFlowTok{in} \DecValTok{1}\OperatorTok{:}\DecValTok{900}\NormalTok{) \{}
\NormalTok{    mu <-}\StringTok{ }\DecValTok{5}
    \CommentTok{# X  <- arima.sim(n = n,  list(ar = c(0.7), ma = c(0.4))  ) + mu                    # Case 1}
\NormalTok{    X  <-}\StringTok{ }\KeywordTok{arima.sim}\NormalTok{(}\DataTypeTok{n =}\NormalTok{ n, }\KeywordTok{list}\NormalTok{(}\DataTypeTok{ar =} \KeywordTok{c}\NormalTok{(}\FloatTok{0.7}\NormalTok{, }\FloatTok{.2}\NormalTok{, }\FloatTok{-.3}\NormalTok{), }\DataTypeTok{ma =} \KeywordTok{c}\NormalTok{(}\FloatTok{0.6}\NormalTok{, }\FloatTok{-.4}\NormalTok{, }\FloatTok{.2}\NormalTok{))  ) }\OperatorTok{+}\StringTok{ }\NormalTok{mu    }\CommentTok{# Case 2}

\NormalTok{    Est      <-}\StringTok{ }\KeywordTok{arima}\NormalTok{(X, }\DataTypeTok{order=}\KeywordTok{c}\NormalTok{(}\DecValTok{3}\NormalTok{,}\DecValTok{0}\NormalTok{,}\DecValTok{3}\NormalTok{))}
\NormalTok{    MLE[i,]  <-}\StringTok{ }\NormalTok{Est}\OperatorTok{$}\NormalTok{coef}
\NormalTok{    Vars[i,] <-}\StringTok{ }\KeywordTok{diag}\NormalTok{(Est}\OperatorTok{$}\NormalTok{var.coef)}
\NormalTok{  \}}
\end{Highlighting}
\end{Shaded}

\begin{Shaded}
\begin{Highlighting}[]
  \CommentTok{#--- Result for phi1, theta1, mu ----}
  \KeywordTok{layout}\NormalTok{(}\KeywordTok{matrix}\NormalTok{(}\DecValTok{1}\OperatorTok{:}\DecValTok{6}\NormalTok{, }\DecValTok{2}\NormalTok{,}\DecValTok{3}\NormalTok{, }\DataTypeTok{byrow=}\OtherTok{TRUE}\NormalTok{)); }
  \KeywordTok{hist}\NormalTok{(MLE[,}\DecValTok{1}\NormalTok{], }\DataTypeTok{main=}\StringTok{"phi1 (.7)"}\NormalTok{);     }\KeywordTok{abline}\NormalTok{(}\DataTypeTok{v=} \FloatTok{.7}\NormalTok{, }\DataTypeTok{col=}\StringTok{"blue"}\NormalTok{)}
  \KeywordTok{hist}\NormalTok{(MLE[,}\DecValTok{2}\NormalTok{], }\DataTypeTok{main=}\StringTok{"phi2 (.2)"}\NormalTok{);     }\KeywordTok{abline}\NormalTok{(}\DataTypeTok{v=} \FloatTok{.2}\NormalTok{, }\DataTypeTok{col=}\StringTok{"blue"}\NormalTok{)}
  \KeywordTok{hist}\NormalTok{(MLE[,}\DecValTok{3}\NormalTok{], }\DataTypeTok{main=}\StringTok{"phi3 (.3)"}\NormalTok{);     }\KeywordTok{abline}\NormalTok{(}\DataTypeTok{v=}\OperatorTok{-}\NormalTok{.}\DecValTok{3}\NormalTok{, }\DataTypeTok{col=}\StringTok{"blue"}\NormalTok{)}
  \KeywordTok{hist}\NormalTok{(MLE[,}\DecValTok{4}\NormalTok{], }\DataTypeTok{main=}\StringTok{"theta1 (.6)"}\NormalTok{);   }\KeywordTok{abline}\NormalTok{(}\DataTypeTok{v=} \FloatTok{.6}\NormalTok{, }\DataTypeTok{col=}\StringTok{"blue"}\NormalTok{)}
  \KeywordTok{hist}\NormalTok{(MLE[,}\DecValTok{5}\NormalTok{], }\DataTypeTok{main=}\StringTok{"theta2 (.4)"}\NormalTok{);   }\KeywordTok{abline}\NormalTok{(}\DataTypeTok{v=}\OperatorTok{-}\NormalTok{.}\DecValTok{4}\NormalTok{, }\DataTypeTok{col=}\StringTok{"blue"}\NormalTok{)}
  \KeywordTok{hist}\NormalTok{(MLE[,}\DecValTok{6}\NormalTok{], }\DataTypeTok{main=}\StringTok{"theta3 (.2)"}\NormalTok{);   }\KeywordTok{abline}\NormalTok{(}\DataTypeTok{v=} \FloatTok{.2}\NormalTok{, }\DataTypeTok{col=}\StringTok{"blue"}\NormalTok{)}
\end{Highlighting}
\end{Shaded}

\includegraphics{TS-Lec53_files/figure-latex/unnamed-chunk-4-1.pdf}

\begin{Shaded}
\begin{Highlighting}[]
\NormalTok{  Result <-}\StringTok{ }\KeywordTok{cbind}\NormalTok{( }\KeywordTok{apply}\NormalTok{(MLE, }\DecValTok{2}\NormalTok{, mean), }\KeywordTok{apply}\NormalTok{(MLE, }\DecValTok{2}\NormalTok{, sd), }\KeywordTok{sqrt}\NormalTok{(}\KeywordTok{apply}\NormalTok{(Vars, }\DecValTok{2}\NormalTok{, mean)) )}
  \KeywordTok{colnames}\NormalTok{(Result) <-}\StringTok{ }\KeywordTok{c}\NormalTok{(}\StringTok{"Mean"}\NormalTok{,}\StringTok{"SE by Sim"}\NormalTok{,  }\StringTok{"SE from Output"}\NormalTok{)}
\NormalTok{  Result}
\end{Highlighting}
\end{Shaded}

\begin{verbatim}
##             Mean SE by Sim SE from Output
## [1,]  0.49674272 0.6293380      0.2109243
## [2,] -0.06994924 0.4727941      0.3126513
## [3,] -0.07590046 0.3371309      0.1800822
## [4,]  0.46670198 0.6569953      0.2178613
## [5,]  0.12501466 0.7223259      0.2789496
## [6,]  0.31208815 0.2956291      0.1982612
## [7,]  4.49790150 1.5393874      0.3230354
\end{verbatim}

\hypertarget{c.-when-errors-are-not-normal}{%
\subsection{c. When errors are not
Normal}\label{c.-when-errors-are-not-normal}}

\begin{Shaded}
\begin{Highlighting}[]
\NormalTok{  n=}\DecValTok{100}
\NormalTok{  MLE  <-}\StringTok{ }\NormalTok{MLE2 <-}\StringTok{ }\NormalTok{Vars <-}\StringTok{ }\NormalTok{Vars2 <-}\StringTok{ }\KeywordTok{matrix}\NormalTok{(}\DecValTok{0}\NormalTok{,}\DecValTok{1000}\NormalTok{,}\DecValTok{3}\NormalTok{)  }\CommentTok{#matrix to record estimated values}
  
  \ControlFlowTok{for}\NormalTok{ (i }\ControlFlowTok{in} \DecValTok{1}\OperatorTok{:}\DecValTok{1000}\NormalTok{) \{}

    \CommentTok{#-- Normal errors ---}
\NormalTok{    X   <-}\StringTok{ }\KeywordTok{arima.sim}\NormalTok{(}\DataTypeTok{n =}\NormalTok{ n, }\KeywordTok{list}\NormalTok{(}\DataTypeTok{ar =} \KeywordTok{c}\NormalTok{(}\FloatTok{0.7}\NormalTok{), }\DataTypeTok{ma =} \KeywordTok{c}\NormalTok{(}\FloatTok{0.4}\NormalTok{))  ) }\OperatorTok{+}\StringTok{ }\NormalTok{mu}
    \CommentTok{#X   <- arima.sim(n = n, list(ar = c(0.7, .2, -.3), ma = c(0.4, -.3, .2))  ) + mu}

\NormalTok{    Est      <-}\StringTok{ }\KeywordTok{arima}\NormalTok{(X, }\DataTypeTok{order=}\KeywordTok{c}\NormalTok{(}\DecValTok{1}\NormalTok{,}\DecValTok{0}\NormalTok{,}\DecValTok{1}\NormalTok{))}
\NormalTok{    MLE[i,]  <-}\StringTok{ }\NormalTok{Est}\OperatorTok{$}\NormalTok{coef}
\NormalTok{    Vars[i,] <-}\StringTok{ }\KeywordTok{c}\NormalTok{(Est}\OperatorTok{$}\NormalTok{var.coef[}\DecValTok{1}\NormalTok{,}\DecValTok{1}\NormalTok{], Est}\OperatorTok{$}\NormalTok{var.coef[}\DecValTok{2}\NormalTok{,}\DecValTok{2}\NormalTok{], Est}\OperatorTok{$}\NormalTok{var.coef[}\DecValTok{3}\NormalTok{,}\DecValTok{3}\NormalTok{])}

    \CommentTok{#-- non-normal errors ---}
\NormalTok{    et <-}\StringTok{ }\KeywordTok{rt}\NormalTok{(n, }\DecValTok{4}\NormalTok{)                          }\CommentTok{# t-distribution with df=4.}
\NormalTok{    Y  <-}\StringTok{ }\KeywordTok{arima.sim}\NormalTok{(}\DataTypeTok{n=}\NormalTok{n, }\KeywordTok{list}\NormalTok{(}\DataTypeTok{ar =} \KeywordTok{c}\NormalTok{(}\FloatTok{0.7}\NormalTok{), }\DataTypeTok{ma =} \KeywordTok{c}\NormalTok{(}\FloatTok{0.4}\NormalTok{)), }\DataTypeTok{innov=}\NormalTok{et  ) }\OperatorTok{+}\StringTok{ }\NormalTok{mu}
    \CommentTok{#Y  <- arima.sim(n = n, list(ar = c(0.7, .2, -.3), ma = c(0.4, -.3, .2))  ) + mu}

\NormalTok{    Est2 <-}\StringTok{ }\KeywordTok{arima}\NormalTok{(Y, }\DataTypeTok{order=}\KeywordTok{c}\NormalTok{(}\DecValTok{1}\NormalTok{,}\DecValTok{0}\NormalTok{,}\DecValTok{1}\NormalTok{))}
\NormalTok{    MLE2[i,]  <-}\StringTok{ }\NormalTok{Est2}\OperatorTok{$}\NormalTok{coef}
\NormalTok{    Vars2[i,] <-}\StringTok{ }\KeywordTok{c}\NormalTok{(Est2}\OperatorTok{$}\NormalTok{var.coef[}\DecValTok{1}\NormalTok{,}\DecValTok{1}\NormalTok{], Est2}\OperatorTok{$}\NormalTok{var.coef[}\DecValTok{2}\NormalTok{,}\DecValTok{2}\NormalTok{], Est2}\OperatorTok{$}\NormalTok{var.coef[}\DecValTok{3}\NormalTok{,}\DecValTok{3}\NormalTok{])}
\NormalTok{  \}}

  \CommentTok{# Plotting Results }
\NormalTok{  p=}\DecValTok{1}    \CommentTok{# chose p: 1=phi1, 2=theta1, 3=mu}

  \KeywordTok{layout}\NormalTok{(}\KeywordTok{matrix}\NormalTok{(}\DecValTok{1}\OperatorTok{:}\DecValTok{4}\NormalTok{, }\DecValTok{2}\NormalTok{,}\DecValTok{2}\NormalTok{, }\DataTypeTok{byrow=}\OtherTok{FALSE}\NormalTok{)); }
  \KeywordTok{hist}\NormalTok{(MLE[,}\DecValTok{1}\NormalTok{],  }\DataTypeTok{main=}\StringTok{"e ~ Norm: phi1 (.7)"}\NormalTok{  );  }\KeywordTok{abline}\NormalTok{(}\DataTypeTok{v=} \FloatTok{.7}\NormalTok{, }\DataTypeTok{col=}\StringTok{"blue"}\NormalTok{) }
  \KeywordTok{hist}\NormalTok{(MLE[,}\DecValTok{2}\NormalTok{],  }\DataTypeTok{main=}\StringTok{"e ~ Norm: theta1 (.4)"}\NormalTok{);  }\KeywordTok{abline}\NormalTok{(}\DataTypeTok{v=} \FloatTok{.4}\NormalTok{, }\DataTypeTok{col=}\StringTok{"blue"}\NormalTok{) }
  \KeywordTok{hist}\NormalTok{(MLE2[,}\DecValTok{1}\NormalTok{], }\DataTypeTok{main=}\StringTok{"e ~ t(v): phi1 (.7)"}\NormalTok{  );  }\KeywordTok{abline}\NormalTok{(}\DataTypeTok{v=} \FloatTok{.7}\NormalTok{, }\DataTypeTok{col=}\StringTok{"blue"}\NormalTok{) }
  \KeywordTok{hist}\NormalTok{(MLE2[,}\DecValTok{2}\NormalTok{], }\DataTypeTok{main=}\StringTok{"e ~ t(v:) theta1 (.4)"}\NormalTok{);  }\KeywordTok{abline}\NormalTok{(}\DataTypeTok{v=} \FloatTok{.4}\NormalTok{, }\DataTypeTok{col=}\StringTok{"blue"}\NormalTok{) }
\end{Highlighting}
\end{Shaded}

\includegraphics{TS-Lec53_files/figure-latex/unnamed-chunk-5-1.pdf}

\begin{Shaded}
\begin{Highlighting}[]
\NormalTok{  Result2 <-}\StringTok{ }\KeywordTok{rbind}\NormalTok{(}\KeywordTok{c}\NormalTok{(}\KeywordTok{mean}\NormalTok{(MLE[,}\DecValTok{1}\NormalTok{]),   }\KeywordTok{sd}\NormalTok{(MLE[,}\DecValTok{1}\NormalTok{]),  }\KeywordTok{mean}\NormalTok{(MLE[,}\DecValTok{2}\NormalTok{]),   }\KeywordTok{sd}\NormalTok{(MLE[,}\DecValTok{2}\NormalTok{] ) ),}
        \KeywordTok{c}\NormalTok{(}\KeywordTok{mean}\NormalTok{(MLE2[,}\DecValTok{1}\NormalTok{]),  }\KeywordTok{sd}\NormalTok{(MLE2[,}\DecValTok{1}\NormalTok{]), }\KeywordTok{mean}\NormalTok{(MLE2[,}\DecValTok{2}\NormalTok{]),  }\KeywordTok{sd}\NormalTok{(MLE2[,}\DecValTok{2}\NormalTok{]) ) )}

  \KeywordTok{colnames}\NormalTok{(Result2) <-}\StringTok{ }\KeywordTok{c}\NormalTok{(}\StringTok{"Mean phi1 (N)"}\NormalTok{,   }\StringTok{"SE phi1 (N)"}\NormalTok{, }\StringTok{"Mean th1 (t)"}\NormalTok{, }\StringTok{"SE th1 (t)"}\NormalTok{)}
\NormalTok{  Result2 }
\end{Highlighting}
\end{Shaded}

\begin{verbatim}
##      Mean phi1 (N) SE phi1 (N) Mean th1 (t) SE th1 (t)
## [1,]     0.6662392  0.09332301    0.4169172  0.1165736
## [2,]     0.6563605  0.09352404    0.4205370  0.1127351
\end{verbatim}

\hypertarget{summary}{%
\section{Summary}\label{summary}}

\hypertarget{cdot-output-for-s.e-of-parameter-estimation-is-calculated-by-using-large-sample-formula.-it-is-not-always-accurate.}{%
\subsubsection{\texorpdfstring{\hspace{10mm} \(\cdot\) Output for s.e of
parameter estimation is calculated by using large-sample formula. It is
not always
accurate.}{ \textbackslash{}cdot Output for s.e of parameter estimation is calculated by using large-sample formula. It is not always accurate.}}\label{cdot-output-for-s.e-of-parameter-estimation-is-calculated-by-using-large-sample-formula.-it-is-not-always-accurate.}}

\hypertarget{cdot-mle-is-derived-assuming-that-e_t-is-normally-distributed-but-when-that-is-violated-performance-of-mle-is-not-significantly-affected.}{%
\subsubsection{\texorpdfstring{\hspace{10mm} \(\cdot\) MLE is derived
assuming that \(e_t\) is Normally distributed, but when that is
violated, performance of MLE is not significantly
affected.}{ \textbackslash{}cdot MLE is derived assuming that e\_t is Normally distributed, but when that is violated, performance of MLE is not significantly affected.}}\label{cdot-mle-is-derived-assuming-that-e_t-is-normally-distributed-but-when-that-is-violated-performance-of-mle-is-not-significantly-affected.}}

\href{https://nmimoto.github.io/477/}{TS Class Web Page} --
\href{https://nmimoto.github.io/R/}{R resource page}

\begin{Shaded}
\begin{Highlighting}[]
  \KeywordTok{ts.plot}\NormalTok{(}\KeywordTok{log}\NormalTok{(Elec), Reg.Fit1, }\DataTypeTok{col=}\KeywordTok{c}\NormalTok{(}\StringTok{"black"}\NormalTok{, }\StringTok{"red"}\NormalTok{), }\DataTypeTok{xlim=}\KeywordTok{c}\NormalTok{(}\DecValTok{1970}\NormalTok{, }\DecValTok{1980}\NormalTok{))}
\end{Highlighting}
\end{Shaded}

\includegraphics{TS-Lec53_files/figure-latex/unnamed-chunk-6-2.pdf}

\begin{Shaded}
\begin{Highlighting}[]
  \KeywordTok{plot}\NormalTok{(Reg.Res1)   }\CommentTok{# plot the Residuals from the regression}
\end{Highlighting}
\end{Shaded}

\includegraphics{TS-Lec53_files/figure-latex/unnamed-chunk-6-3.pdf}

\begin{Shaded}
\begin{Highlighting}[]
  \KeywordTok{library}\NormalTok{(forecast)                                            }\CommentTok{# load forecast package}
  \KeywordTok{source}\NormalTok{(}\StringTok{'https://nmimoto.github.io/R/TS-00.txt'}\NormalTok{)              }\CommentTok{# load Randomness.tests}
  

\NormalTok{  Fit1 <-}\StringTok{ }\KeywordTok{auto.arima}\NormalTok{(Reg.Res1, }\DataTypeTok{d=}\DecValTok{0}\NormalTok{, }\DataTypeTok{stepwise=}\OtherTok{FALSE}\NormalTok{)}
\NormalTok{  Fit2 <-}\StringTok{ }\KeywordTok{Arima}\NormalTok{(Reg.Res1, }\DataTypeTok{order=}\KeywordTok{c}\NormalTok{(}\DecValTok{3}\NormalTok{,}\DecValTok{0}\NormalTok{,}\DecValTok{7}\NormalTok{))}

  \KeywordTok{Randomness.tests}\NormalTok{(Fit1}\OperatorTok{$}\NormalTok{resid)}
\end{Highlighting}
\end{Shaded}

\includegraphics{TS-Lec53_files/figure-latex/unnamed-chunk-6-4.pdf}

\begin{verbatim}
##   B-L test H0: the sereis is uncorrelated
##   M-L test H0: the square of the sereis is uncorrelated
##   J-B test H0: the sereis came from Normal distribution
##   SD         : Standard Deviation of the series
\end{verbatim}

\begin{verbatim}
##       BL15  BL20  BL25  ML15  ML20    JB   SD
## [1,] 0.232 0.058 0.003 0.458 0.206 0.001 0.02
\end{verbatim}

\begin{center}\rule{0.5\linewidth}{\linethickness}\end{center}

Lake HURON

\begin{Shaded}
\begin{Highlighting}[]
\CommentTok{#--- Analysis 1: direct fit}

  \KeywordTok{library}\NormalTok{(forecast)}
  \KeywordTok{source}\NormalTok{(}\StringTok{"http://gozips.uakron.edu/~nmimoto/477/TS_R-90.txt"}\NormalTok{)}
  
\NormalTok{  X1 <-}\StringTok{ }\KeywordTok{read.csv}\NormalTok{(}\StringTok{"http://gozips.uakron.edu/~nmimoto/pages/datasets/lake.txt"}\NormalTok{)}
\NormalTok{  X  <-}\StringTok{ }\KeywordTok{ts}\NormalTok{(X1, }\DataTypeTok{start=}\DecValTok{1875}\NormalTok{, }\DataTypeTok{freq=}\DecValTok{1}\NormalTok{)}
  
  \KeywordTok{plot}\NormalTok{(X, }\DataTypeTok{type=}\StringTok{"o"}\NormalTok{)}
  
  \KeywordTok{layout}\NormalTok{(}\KeywordTok{matrix}\NormalTok{(}\KeywordTok{c}\NormalTok{(}\DecValTok{1}\NormalTok{,}\DecValTok{2}\NormalTok{), }\DecValTok{1}\NormalTok{, }\DecValTok{2}\NormalTok{)}
  \KeywordTok{acf}\NormalTok{(X);  }\KeywordTok{pacf}\NormalTok{(X); }\KeywordTok{layout}\NormalTok{(}\DecValTok{1}\NormalTok{)}
  
\NormalTok{  Fit1 <-}\StringTok{ }\KeywordTok{auto.arima}\NormalTok{(X, }\DataTypeTok{d=}\DecValTok{0}\NormalTok{);   Fit1    }\CommentTok{# find best ARMA(p,q) by AICc}
  
  \KeywordTok{plot}\NormalTok{(Fit1}\OperatorTok{$}\NormalTok{residuals)}
  
  \KeywordTok{layout}\NormalTok{(}\KeywordTok{matrix}\NormalTok{(}\KeywordTok{c}\NormalTok{(}\DecValTok{1}\NormalTok{,}\DecValTok{2}\NormalTok{), }\DecValTok{1}\NormalTok{, }\DecValTok{2}\NormalTok{))}
  \KeywordTok{acf}\NormalTok{(Fit1}\OperatorTok{$}\NormalTok{residuals);  }\KeywordTok{pacf}\NormalTok{(Fit1}\OperatorTok{$}\NormalTok{residuals); }\KeywordTok{layout}\NormalTok{(}\DecValTok{1}\NormalTok{)}
  
  \KeywordTok{Randomness.tests}\NormalTok{(Fit1}\OperatorTok{$}\NormalTok{residuals)}
\end{Highlighting}
\end{Shaded}

Analysis 1 (direct fit)

\begin{itemize}
\item
  auto.arima() chooses AR(2) with min AICC.
\item
  AR(2) with constant mean was fit directly to data. \[
  Y_t \hspace3mm = \hspace3mm \mu + X_t \\\\
  X_t \hspace3mm = \hspace3mm \phi_1 X_{t-1} + \phi_2 X_{t-2} + e_t 
  \]
\end{itemize}

---------------------------------------------\textgreater{}


\end{document}
